\documentclass[pdftex]{beamer}
\usetheme{metropolis}

\usepackage[english]{babel}
\usepackage[latin1]{inputenc}
\usepackage{times}
\usepackage[T1]{fontenc}
\usepackage{fancyvrb}
\usepackage{listings}
\begin{document}
\lstset{language=C, escapeinside={(*@}{@*)}, numbers=left,
  basicstyle=\tiny, showspaces=false, showtabs=false, showstringspaces=false}

\title{A Look Inside FreeBSD with DTrace}
\subtitle{Networking}
\author[shortname]{George V. Neville-Neil \and Robert N. M. Watson}

\begin{frame}
  \frametitle{Networking and FreeBSD}
  \begin{itemize}
  \item Everyone's TCP/IP Stack
  \item IPv4, IPv6, UDP, TCP, SCTP
  \item Various drivers
  \item Multiple firewalls
  \end{itemize}
\end{frame}

\begin{frame}
  \frametitle{The User Program View}
  \begin{itemize}
  \item User programs use sockets
  \item Network programs follow UNIX model
  \item Flexible interfaces for different protocols
  \end{itemize}
\end{frame}

\begin{frame}
  \frametitle{Sockets}
  \begin{itemize}
  \item Main programmer interface to networking
  \item Generic API
  \item Attempts to support read/write semantics
  \end{itemize}
\end{frame}

\begin{frame}[fragile]
  \frametitle{Looking Directly at Sockets}
\begin{verbatim}
# Count sockets by family

# Count sockets by type

# Count sockets by protocol
\end{verbatim}  
\end{frame}

\begin{frame}
  \frametitle{Network Lab (Sockets Exercises)}
  \begin{itemize}
  \item Count socket calls by domain, type and protocol
  \item Show programs accepting connections
  \item Show programs initiating connections
  \item Write a D script to trace a single socket with the test program
  \end{itemize}
\end{frame}

\begin{frame}
  \frametitle{Network Stack}
\end{frame}

\begin{frame}
  \frametitle{Network Stack Overview}
\centering
\includegraphics{\FigPath/tcpipstack}
\end{frame}

\begin{frame}
\centering
  \frametitle{Inbound Layer Transitions}
\includegraphics[width=0.5\textwidth]{\FigPath/inbound}
\end{frame}

\begin{frame}[fragile]
  \frametitle{UDP}
  \begin{itemize}
  \item Simplest transport protocol
  \item No states to maintain
  \item Data is sent immediately
  \item Supports multicast
  \item Only probes are \verb+send+ and \verb+receive+
  \end{itemize}
\end{frame}

\begin{frame}[fragile]
  \frametitle{UDP Send and Receive}
  \begin{itemize}
  \item \verb|udptrack|
  \end{itemize}
\end{frame}

\begin{frame}[fragile]
  \frametitle{Network Protocol Lab Exercises}
  \begin{itemize}
  \item Add IP source and destination information to \verb+tcpstate+
  \item Add support for \verb+send+ and \verb+receive+ calls to \verb+tcptrack+
  \item Show the congestion window for a single connection over time
  \end{itemize}
\end{frame}

\begin{frame}
  \frametitle{Packet Forwarding}
  \begin{itemize}
  \item System as a router, switch or firewall
  \item Network Layer Packets only
  \end{itemize}
\end{frame}

\begin{frame}
  \frametitle{A Worked Example}
      \includegraphics[width=\textwidth]{\FigPath/labsetup.png}
\end{frame}

\begin{frame}[fragile]
  \frametitle{Forward vs. Fast Forward}
  \begin{itemize}
  \item What difference does this make?
    \begin{itemize}
    \item \verb+net.inet.ip.fastforward+
    \end{itemize}
  \item Where do we look?
  \item What can be known?
  \end{itemize}
\end{frame}

\begin{frame}[fragile]
  \frametitle{Normal vs. Fast}
\begin{verbatim}
value  ------------- Distribution ------------- count    
512 |                                         0        
1024 |@@@@@@@@@@@@@@@@@@@@@@@  1414505  
2048 |@                                        35478    
4096 |                                         481      
8192 |                                         0        
\end{verbatim}

\begin{verbatim}
 value  ------------- Distribution ------------- count    
512 |                                         0        
1024 |@@@@@@@@@@@@@@@@@@@@  1721837  
2048 |@                                        41287    
4096 |                                         490      
8192 |                                         0        
\end{verbatim}
\end{frame}


\begin{frame}[fragile]
  \frametitle{Network Lab (Protocols)}
  \begin{itemize}
  \item Show inbound connections to sshd
  \item What routines are called when a ping packet arrives?
  \item What routines are called before \verb+tcp_output()+?
  \end{itemize}
\end{frame}

\end{document}

%%% Local Variables:
%%% mode: latex
%%% TeX-master: t
%%% End:
