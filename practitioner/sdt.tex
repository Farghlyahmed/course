\documentclass[pdftex]{beamer}
\usetheme{metropolis}

\usepackage[english]{babel}
\usepackage[latin1]{inputenc}
\usepackage{times}
\usepackage[T1]{fontenc}
\usepackage{fancyvrb}
\usepackage{listings}
\begin{document}
\lstset{language=C, escapeinside={(*@}{@*)}, numbers=left,
  basicstyle=\tiny, showstringspaces=false, showspaces=false, showtabs=false}

\title{A Look Inside FreeBSD with DTrace}
\subtitle{Kernel SDTs}
\author[shortname]{George V. Neville-Neil \and Robert N. M. Watson}

\begin{frame}
  \titlepage
\end{frame}

\begin{frame}[fragile]
  \frametitle{Converting Logging Code}
  \begin{itemize}
  \item Most code littered with \verb|printf|
  \item Many different \verb|DEBUG| options
  \item  Most can be converted
  \end{itemize}
\end{frame}

\begin{frame}[fragile]
  \frametitle{TCPDEBUG Case Study}
  \begin{itemize}
  \item TCBDEBUG added in the original BSD releases
  \item Rarely enabled kernel option that shows:
    \begin{itemize}
    \item direction
    \item state
    \item sequence space
    \item \verb|rcv_nxt|, \verb|rcv_wnd|, \verb|rcv_up|
    \item \verb|snd_una|, \verb|snd_nxt|, \verb|snx_max|
    \item \verb|snd_wl1|, \verb|snd_wl2|, \verb|snd_wnd|
    \end{itemize}
  \end{itemize}
\end{frame}

\begin{frame}
  \frametitle{TCPDEBUG Before}
  \begin{itemize}
  \item $127$ lines of code
  \item $14$ calls to printf
  \item Statically defined ring buffer of $100$ entries
  \item Static log format
  \end{itemize}
\end{frame}

\begin{frame}
  \frametitle{TCPDEBUG After}
  \begin{itemize}
  \item Four (4) new tracepoints
    \begin{itemize}
    \item debug-input
    \item debug-output
    \item debug-user
    \item debug-drop
    \end{itemize}
  \item Access to TCP and socket structures
  \item Flexible log format
  \end{itemize}
\end{frame}

\begin{frame}[fragile]
  \frametitle{Convenient Macros}
  \begin{itemize}
  \item \verb|SDT_PROVIDER_DECLARE| Declare a provider in an include file
  \item \verb|SDT_PROVIDER_DEFINE| Instantiate a provider in C code
  \item \verb|SDT_PROBE_DECLARE| Declare a probe in a n include file
  \item \verb|SDT_PROBE_DEFINEN| Define a probe of X arguments (0-6)
  \item \verb|SDT_PROBE_DEFINEN_XLATE| Define a probe of N arguments with translation
  \item Only available for kernel code
  \end{itemize}
\end{frame}

\begin{frame}[fragile]
  \frametitle{TCP Debug Desclarations}
\begin{lstlisting}
SDT_PROBE_DECLARE(tcp, , , debug__input);
SDT_PROBE_DECLARE(tcp, , , debug__output);
SDT_PROBE_DECLARE(tcp, , , debug__user);
SDT_PROBE_DECLARE(tcp, , , debug__drop);
\end{lstlisting}
\end{frame}

\begin{frame}[fragile]
  \frametitle{TCP Debug Call Sites}
\begin{lstlisting}
#ifdef TCPDEBUG
	if (tp == NULL || (tp->t_inpcb->inp_socket->so_options & SO_DEBUG))
		tcp_trace(TA_DROP, ostate, tp, (void *)tcp_saveipgen,
			  &tcp_savetcp, 0);
#endif
	TCP_PROBE3(debug__input, tp, th, mtod(m, const char *));
\end{lstlisting}
\end{frame}

\begin{frame}[fragile]
  \frametitle{TCP Debug Translators}
\begin{lstlisting}
SDT_PROBE_DEFINE3_XLATE(tcp, , , debug__input,
    "struct tcpcb *", "tcpsinfo_t *" ,
    "struct tcphdr *", "tcpinfo_t *",
    "uint8_t *", "ipinfo_t *");

SDT_PROBE_DEFINE3_XLATE(tcp, , , debug__output,
    "struct tcpcb *", "tcpsinfo_t *" ,
    "struct tcphdr *", "tcpinfo_t *",
    "uint8_t *", "ipinfo_t *");

SDT_PROBE_DEFINE2_XLATE(tcp, , , debug__user,
    "struct tcpcb *", "tcpsinfo_t *" ,
    "int", "int");

SDT_PROBE_DEFINE3_XLATE(tcp, , , debug__drop,
    "struct tcpcb *", "tcpsinfo_t *" ,
    "struct tcphdr *", "tcpinfo_t *",
    "uint8_t *", "ipinfo_t *");
\end{lstlisting}
\end{frame}

\begin{frame}[fragile]
  \frametitle{TCP Debug Example Script}
\begin{lstlisting}
tcp:kernel::debug-input
/args[0]->tcps_debug/
{
	seq = args[1]->tcp_seq;
	ack = args[1]->tcp_ack;
	len = args[2]->ip_plength - sizeof(struct tcphdr);
	flags = args[1]->tcp_flags;
	
	printf("%p %s: input [%xu..%xu]", arg0,
	       tcp_state_string[args[0]->tcps_state], seq, seq + len);

	printf("@%x, urp=%x", ack, args[1]->tcp_urgent);
\end{lstlisting}
\end{frame}

\begin{frame}[fragile]
  \frametitle{TCP Debug Example Script Part 2}
\begin{lstlisting}
	printf("%s", flags != 0 ? "<" : "");
	printf("%s", flags & TH_SYN ? "SYN," :"");
	printf("%s", flags & TH_ACK ? "ACK," :"");
	printf("%s", flags & TH_FIN ? "FIN," :"");
	printf("%s", flags & TH_RST ? "RST," :"");
	printf("%s", flags & TH_PUSH ? "PUSH," :"");
	printf("%s", flags & TH_URG ? "URG," :"");
	printf("%s", flags & TH_ECE ? "ECE," :"");
	printf("%s", flags & TH_CWR ? "CWR" :"");
	printf("%s", flags != 0 ? ">" : "");

	printf("\n");
	printf("\trcv_(nxt,wnd,up) (%x,%x,%x) snd_(una,nxt,max) (%x,%x,%x)\n",
	       args[0]->tcps_rnxt, args[0]->tcps_rwnd, args[0]->tcps_rup,
	       args[0]->tcps_suna, args[0]->tcps_snxt, args[0]->tcps_smax);
	printf("\tsnd_(wl1,wl2,wnd) (%x,%x,%x)\n",
	       args[0]->tcps_swl1, args[0]->tcps_swl2, args[0]->tcps_swnd);
\end{lstlisting}
\end{frame}

\begin{frame}
  \frametitle{How Much Work is That?}
  \begin{itemize}
  \item $200$ line code change
  \item $167$ lines of example code
  \item A few hours to code
  \item A day or two to test
  \item Now we have always on TCP debugging
  \end{itemize}
\end{frame}

\begin{frame}
  \frametitle{Lab Exercise: Adding Kernel Tracepoints}
  
\end{frame}

\end{document}

%%% Local Variables:
%%% mode: latex
%%% TeX-master: t
%%% End:
