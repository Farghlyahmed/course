\documentclass[a4paper,10pt]{article}
\usepackage{fullpage}
\usepackage{url}
\usepackage{times}

\begin{document}
\title{L41: Readings}
\author{Dr Robert N.M. Watson}
\date{Michaelmas Term 2016}
\maketitle

\section*{Reading assignments}
Reading assignments should be completed prior to the lecture or lab that they
correspond to.
Full citations for books and papers may be found below.

\begin{description}
\item[Lecture 1:]

  McKusick, et al: Chapter 2 (\textit{Design Overview of FreeBSD}).

\item[Lecture 2:]

  McKusick, et al: Chapter 3 (\textit{Kernel Subsystems}).

  Paper - Cantrill, et al. 2004.

\item[Lab 1:]

  Gregg and Mauro: Chapters 1 (\textit{Introduction to DTrace}) and
  2 (\textit{D Language}).

\item[Lecture 3:]

  McKusick, et al: Chapter 4 (\textit{Process Management}).

  Paper - Anderson, et al. 1992.

\item[Lecture 4:]

  McKusick, et al: Chapter 6 (\textit{Memory Management}).

  Paper - Navarro, et al. 2002.

\item[Lab 2:]

  Paper - Ellard and Seltzer 2003.

\item[Lab 3:]

  % XXXRW: For next year -- would be nice to have a chapter to read on the
  % topic of microarchitecture and counters .. perhaps the chapter from
  % Brendan's other book on cloud performance..?

  \textit{No reading assignment}.

\item[Lecture 5:]

  McKusick, et al: Chapter 12 (\textit{Inter-Process Communication}).

  Paper - Rizzo 2012.

\item[Lecture 6:]

  McKusick, et al: Chapter 14 (\textit{Transport-Layer Protocols}).

  Paper - Marinos, et al. 2014.

\item[Lab 4:]

  Paper - Bishop, et al. 2005.

\item[Lab 5:]

  Paper - Van Jacobson 1988.
\end{description}

\section*{Course texts}
Course texts provide instruction on statistics, operating-system design and
implementation, and system tracing.
You will be asked to read selected chapters from these, but will likely find
other content in them useful as you proceed with the labs.

\medskip

\noindent
Marshall Kirk McKusick, George V. Neville-Neil, and Robert N. M. Watson.
\textit{The Design and Implementation of the FreeBSD Operating System, 2nd
Edition}, Pearson Education, Boston, MA, USA, September 2014.


\medskip
\noindent
Brendan Gregg and Jim Mauro. \textit{DTrace: Dynamic Tracing in Oracle
Solaris, Mac OS X and FreeBSD}, Prentice Hall Press, Upper Saddle River, NJ,
USA, April 2011.

\medskip
\noindent
Raj Jain, \textit{The Art of Computer Systems Performance Analysis: Techniques
for Experimental Design, Measurement, Simulation, and Modeling}, Wiley -
Interscience, New York, NY, USA, April 1991.

\section*{Research readings}

Our research readings are drawn from various systems publications venues;
these provide insight into types of research done with systems that are
particularly relevant to our laboratory work, but also examples of practical
systems research.
Some readings are assigned prior to specific lectures or labs; others are for
your (optional) enlightenment (and hopefully also enjoyment).

\subsection*{Tracing and performance analysis}

Bryan M. Cantrill, Michael W. Shapiro and Adam H. Leventhal.  \textit{Dynamic
Instrumentation of Production Systems}.  Proceedings of the 2004 USENIX Annual
Technical Conference, USENIX Association, June, 2004.

\medskip
\noindent
Daniel Ellard and Margo Seltzer.  \textit{NFS Tricks and Benchmarking Traps}.
Proceedings of the 2003 USENIX Annual Technical Conference, FREENIX Track,
USENIX Association, June, 2003.

\medskip
\noindent
Luigi Rizzo.  \textit{Dummynet: A Simple Approach to the Evaluation of Network
Protocols}, ACM SIGCOMM Computer Communication Review 27(1), 31-41, ACM, 1997.
\textbf{(Optional reading)}

\subsection*{Kernel structure and primitives}

Mike Accetta, Robert Baron, William Bolosky, David Golub, Richard Rashid,
Avadis Tevanian, and Michael Young. \textit{Mach: A New Kernel Foundation for
UNIX Development}.  Proceedings of the 1986 USENIX Summer Conference, USENIX
Association, June, 1986.  \textbf{(Optional reading)}

\medskip
\noindent
Thomas E. Anderson, Brian N. Bershad, Edward D. Lazowska, and Henry M. Levy.
\textit{Scheduler Activations: Effective Kernel Support for User-Level
Management of Parallelism}. ACM Transactions on Computer Systems, 10(1),
53-79, ACM, February 1992.

\medskip
\noindent
Juan Navarro, Sitaram Iyer, Peter Druschel, Alan L. Cox.  \textit{Practical,
Transparent Operating System Support for Superpages}.  5th Symposium on
Operating Systems Design and Implementation (OSDI '02), USENIX Association,
December, 2002.

\medskip
\noindent
Paul Barham, Boris Dragovic, Keir Fraser, Steven Hand, Tim Harris, Alex Ho,
Rolf Neugebauer, Ian Pratt, and Andrew Warfield.  \textit{Xen and the Art of
Virtualization}. Proceedings of the 19th ACM Symposium on Operating Systems
Principles (SOSP'03), ACM, October 2003.  \textbf{(Optional reading)}

\medskip
\noindent
Silas Boyd-Wickizer, Austin T. Clements, Yandong Mao, Aleksey Pesterev, M.
Frans Kaashoek, Robert Morris, and Nickolai Zeldovich.  \textit{An Analysis of
Linux Scalability to Many Cores}.  Proceedings of the 9th USENIX Symposium on
Operating System Design and Implementation (OSDI '10), USENIX Association,
October, 2010.  \textbf{(Optional reading)}

\subsection*{Network stacks}

Steve Bishop, Matthew Fairbairn, Michael Norrish, Peter Sewell, Michael Smith,
and Keith Wansbrough.  \textit{Rigorous Specification and Conformance Testing
Techniques for Network Protocols, as Applied to TCP, UDP, and Sockets}.
Proceedings of SIGCOMM 2005, ACM, 2005.

\medskip
\noindent
Van Jacobson.  \textit{Congestion avoidance and control}.  Proceedings of
SIGCOMM, ACM, 1988.

\medskip
\noindent
Steven McCanne and Van Jacobson.  \textit{The BSD Packet Filter: A New
Architecture for User-level Packet Capture}.  Proceedings of the 1993 USENIX
Winter Conference, USENIX Association, January, 1993.  \textbf{(Optional
reading)}

\medskip
\noindent
Luigi Rizzo.  \textit{netmap: A Novel Framework for Fast Packet I/O}.
Proceedings of the USENIX 2012 Annual Technical Conference (ATC'12), USENIX
Association, June, 2012.

\medskip
\noindent
Ilias Marinos, Robert N. M. Watson, and Mark Handley, \textit{Network Stack
Specialization for Performance}, Proceedings of SIGCOMM, ACM, August, 2014.

\section*{Supplemental course texts}

The supplemental readings may be useful in refreshing or building up your
basic knowledge and skills in support of our lectures and labs.

\medskip

\noindent
Abraham Silberschatz, Peter Baer Galvin, and Greg Gagne, \textit{Operating
System Concepts, Eighth Edition}, John Wiley and Sons, Inc., New York, NY, USA,
July 2008.

\medskip
\noindent
Brendan Gregg. \textit{Systems Performance: Enterprise and the Cloud},
Prentice Hall Press, Upper Saddle River, NJ, USA, October 2013.

\medskip
\noindent
Wes McKinney, \textit{Python for Data Analysis: Data Wrangling with Pandas,
NumPy, and IPython}, O'Reilly Media, Sebastopol, CA, USA, October 2012.

\section*{Websites}

These websites may also be of use:

\bigskip

\begin{tabular}{ll}
L41 Module Page & \url{https://www.cl.cam.ac.uk/teaching/1617/L41/} \\
FreeBSD Project & \url{https://www.FreeBSD.org/} \\
FreeBSD Subversion Repository & \url{https://svn.FreeBSD.org/} \\
DTrace on FreeBSD & \url{https://wiki.freebsd.org/DTrace} \\
FreeBSD and Linux Kernel Cross-Reference & \url{http://fxr.watson.org/} \\
FreeBSD Benchmark Advice & \url{https://wiki.freebsd.org/BenchmarkAdvice} \\
BeagleBone Black & \url{http://beagleboard.org/black} \\
\end{tabular}

\end{document}
