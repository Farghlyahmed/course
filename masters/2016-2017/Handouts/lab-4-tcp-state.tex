\documentclass[a4paper,10pt]{article}

\usepackage{fullpage}
\usepackage{times}
\usepackage{url}

\begin{document}

\title{L41: Lab 4 - The TCP State Machine}
\author{Dr Robert N.M. Watson \and Dr Graeme Jenkinson}
\date{Lent Term 2017}
\maketitle

\noindent
The goals of this lab are to:

\begin{itemize}
\item Use DTrace to investigate the actual TCP state machine and its
  interactions with the network stack
\item Use DTrace and DUMMYNET to investigate the effects of latency on TCP
  state transitions
\end{itemize}

\noindent
In this lab, we begin that investigation, which will be extended to include
additional exploration of TCP bandwidth, as affected by latency, in the
following lab.

\section*{Background: Transmission Control Protocol (TCP)}

The Transmission Control Protocol (TCP) is a near-universally used protocol
that provides reliable, bi-directional, ordered byte streams over the Internet
Protocol (IP) between two communication endpoints.
TCP connections are built between a pair of IP addresses, identifying host
network interfaces, and port numbers selected applications (or automatically
by the kernel) on either endpoint -- collectively, a \textit{4-tuple}.
While other models are possible, typical TCP use has one side play the role of
a `server', which provides some network-reachable service on a
\textit{well-known port}, and the other the `client', building a connection to
reach that service from an \textit{ephemeral port} randomly selected by the
client TCP implementation.

The BSD (and now POSIX) sockets API offers a portable and simple interface for
TCP/IP client and server programming.
The server opens a socket using the \texttt{socket(2)} system call, binds a
well-known or previously negotiated port number using \texttt{bind(2)}, and
performs \texttt{listen(2)} to begin accepting new connections, returned as
additional connected sockets from calls to \texttt{accept(2)}.
The client application similarly calls \texttt{socket(2)} to open a socket,
and \texttt{connect(2)} to connect to a target address and port number.
Once open, both sides can use system calls such as \texttt{read(2)},
\texttt{write(2)}, \texttt{send(2)}, and \texttt{recv(2)} to send and receive
data over the connection.
The \texttt{close(2)} system call both initiates a connection close (if not
already closed) and releases the socket -- whose state may persist for some
further period to allow data to drain and prevent premature re-use of the
4-tuple.

As discussed in lecture, TCP connections are implemented by a pair of state
machine instances, one on each communications endpoint.
Once in the \texttt{ESTABLISHED} steady state, data passes in each direction
via \textit{segments} that are \textit{acknowledged} by packets passing in the
other direction.
The rate of data flow is controlled by TCP's \textit{flow-control} and
\textit{congestion-control} mechanisms that respectively prevent the sender
from sending more data than the receiver or network can handle.
Congestion control operates in three phases: \textit{slow start}, in which use
of bandwidth is rapidly ramped up to exploit available network bandwidth
either at the start of the connection or following a timeout, and two tightly
coupled phases of \textit{congestion avoidance} and \textit{fast recovery} as
TCP discovers and maintains a \textit{congestion window} close to available
fair bandwidth limit.

TCP identifies every byte in one direction of a connection via a sequence
number.
Data segments contain a starting sequence number and length, describing the
range of transmitted bytes.
Acknowledgment packets contain the sequence number of the byte that follows
the last contiguous byte they are acknowledging.
Acknowledgments are piggybacked onto data segments traveling in the opposite
direction to the greatest extent possible to avoid additional packet
transmissions.
In slow start, TCP performance is directly limited by latency, as the
congestion window can be opened only by receiving \texttt{ACKs} -- which
require successive round trips.
These periods are referred to as \textit{latency bound} for this reason, and
network latency a critical factor in effective utilisation of path bandwidth.

\section*{The benchmark}

Our IPC benchmark also supports a \texttt{tcp\_socket} IPC
type which requests use of TCP over the \textit{loopback interface} on port
\texttt{10141}.
Use of a fixed port number makes it easy to identify and classify experimental
packets on the loopback interface using packet-sniffing tools such as
\texttt{tcpdump}, and also via DTrace predicates.
You are advised to minimise network activity during the running of TCP-related
benchmarks, and when using DTrace, to reduce the degree of interference both
from the perspective of analysing behaviour, and for reasons of the probe
effect.

\subsection*{Compiling the benchmark}

Labs 4 and 5 use the same IPC benchmark utilized in Labs 2 and 3.
Follow the instructions present in those lab assignments to build and use the
IPC benchmark.

\subsection*{Running the benchmark}

As before, you can run the benchmark using the \texttt{ipc-static} and
\texttt{ipc-dynamic} commands, specifying various benchmark parameters.
For the purposes of this benchmark, we recommend the following configuration:

\begin{itemize}
\item Use \texttt{ipc-static}
\item Use 2-thread mode
\item Do not set the socket-buffer size flag
\item Do not modify the total I/O size
\end{itemize}

\noindent
Do ensure that, as in Labs 2 and 3, you have increased the kernel's maximum 
socket-buffer size.

\subsection*{IPFW and DUMMYNET}

To control latency for our experimental traffic, we will employ the IPFW
firewall for packet classification, and the DUMMYNET traffic-control facility
to pass packets over simulated `pipes'.
To configure two 1-way DUMMYNET pipes, each carrying a 10ms one-way latency,
run the following commands as root:

\begin{verbatim}
ipfw pipe config 1 delay 10
ipfw pipe config 2 delay 10
\end{verbatim}

\noindent
During your experiments, you will wish to change the simulated latency to
other values, which can be done by reconfiguring the pipes.
Do this by repeating the above two commands but with modified last parameters,
which specify one-way latencies in milliseconds (e.g., replace `10' with `5'
in both commands).
The total Round-Trip Time (RTT) is the sum of the two latencies -- i.e., 10ms
in each direction comes to a total of 20ms RTT.
Note that DUMMYNET is a simulation tool, and subject to limits on granularity
and precision.
Next, you must assign traffic associated with the experiment, classified by
its TCP port number and presence on the loopback interface (\texttt{lo0}), to
the pipes to inject latency:

\begin{verbatim}
ipfw add 1 pipe 1 tcp from any 10141 to any via lo0
ipfw add 2 pipe 2 tcp from any to any 10141 via lo0
\end{verbatim}

\noindent
You should configure these firewall rules only once per boot.

\subsection*{Configuring the loopback MTU}

Network interfaces have a configured Maximum Transmission Unit (MTU) -- the
size, in bytes, of the largest packet that can be sent.
For most Ethernet and Ethernet-like interfaces, the MTU is typically 1,500
bytes, although larger `jumbograms' can also be used in LAN environments.
The loopback interface provides a simulated network interface carrying
traffic for loopback addresses such as 127.0.0.1 (\texttt{localhost}), and
typically uses a larger (16K+) MTU.
To allow our simulated results to more closely resemble LAN or WAN traffic,
run the following command as root to set the loopback-interface MTU to 1,500
bytes after each boot:

\begin{verbatim}
ifconfig lo0 mtu 1500
\end{verbatim}

\subsection*{Example benchmark command}

This command instructs the IPC benchmark to perform a transfer over TCP in
2-thread mode:

\begin{verbatim}
ipc/ipc-static -v -i tcp 2thread
\end{verbatim}

\section*{DTrace probes for TCP}

FreeBSD's DTrace implementation contains a number of probes pertinent to TCP,
which you may use in addition to system-call and other probes you have
employed in prior labs:

\begin{description}
\item[fbt::syncache\_add:entry] FBT probe when a \texttt{SYN} packet is
  received for a listening socket, which will lead to a SYN cache entry being
  created.
  The third argument (\texttt{args[2]}) is a pointer to a \texttt{struct
  tcphdr}.

\item[fbt::syncache\_expand:entry] FBT probe when a TCP packet converts a
  pending SYN cookie or SYN cache connection into a full TCP connection.
  The third argument (\texttt{args[2]}) is a pointer to a \texttt{struct
  tcphdr}.

\item[fbt::tcp\_do\_segment:entry] FBT probe when a TCP packet is received in
  the `steady state'.
  The second argument (\texttt{args[1]}) is a pointer to a \texttt{struct
  tcphdr} that describes the TCP header (see RFC 893).
  You will want to classify packets by port number to ensure that you are
  collecting data only from the flow of interest (port \texttt{10141}), and
  associating collected data with the right direction of the flow.
  Do this by checking TCP header fields \texttt{th\_sport} (source port) and
  \texttt{th\_dport} (destination port) in your DTrace predicate.
  In addition, the fields \texttt{th\_seq} (sequence number in transmit
  direction), \texttt{th\_ack} (ACK sequence number in return direction), and
  \texttt{th\_win} (TCP advertised window) will be of interest.
  The fourth argument (\texttt{args[3]}) is a pointer to a \texttt{struct
  tcpcb} that describes the active connection.

\item[fbt::tcp\_state\_change:entry] FBT probe that fires when a TCP state
  transition takes place.
  The first argument (\texttt{args[0]}) is a pointer to a \texttt{struct
  tcpcb} that describes the active connection.
  The \texttt{tcpcb} field \texttt{t\_state} is the previous state of the
  connection.
  Access to the connection's port numbers at this probe point can be
  achieved by following \texttt{t\_inpcb->inp\_inc.inc\_ie}, which
  has fields \texttt{ie\_fport} (foreign, or remote port) and
  \texttt{ie\_lport} (local port) for the connection.
  The second argument is the new state to be assigned.
\end{description}

When analysing TCP states, the D array \texttt{tcp\_state\_string} can be used
to convert an integer state to a human-readable string (e.g., 0 to
\texttt{TCPS\_CLOSED}).
For these probes, the port number will be in \textit{network byte order}; the
D function \texttt{ntohs()} can be used to convert to host byte order when
printing or matching values in \texttt{th\_sport}, \texttt{th\_dport},
\texttt{ie\_lport}, and \texttt{ie\_fport}.
Note that sequence and acknowledgment numbers are cast to unsigned integers.
When analysing and graphing data, be aware that sequence numbers can (and
will) wrap due to the 32-bit sequence space.

\section*{Sample DTrace scripts}

The following script prints out, for each received TCP segment beyond the
initial SYN handshake, the sequence number, ACK number, and state of the TCP
connection prior to full processing of the segment:

\begin{verbatim}
dtrace -n 'fbt::tcp_do_segment:entry {
  trace((unsigned int)args[1]->th_seq);
  trace((unsigned int)args[1]->th_ack);
  trace(tcp_state_string[args[3]->t_state]);
}'
\end{verbatim}

\noindent
Trace state transitions printing the receiving and sending port numbers for
the connection experiencing the transition:

\begin{verbatim}
dtrace -n fbt::tcp_state_change:entry '{
  trace(ntohs(args[0]->t_inpcb->inp_inc.inc_ie.ie_lport));
  trace(ntohs(args[0]->t_inpcb->inp_inc.inc_ie.ie_fport));
  trace(tcp_state_string[args[0]->t_state]);
  trace(tcp_state_string[args[1]]);
}'
\end{verbatim}

\noindent
These scripts can be extended to match flows on port \texttt{10141} in either
direction as needed.

\section*{Experimental questions (part 1)}

These questions form the first part of your lab report spanning Labs 4 and 5.
As described above, configure the IPC benchmark to use TCP in \texttt{2thread}
mode.
When exploring TCP state-machine behaviour, use whole-program analysis.
When exploring the effects of latency on performance, use only I/O-loop
analysis.
Employ the Graphviz tool to plot state machines automatically from
measurements captured using DTrace scripts.

\begin{enumerate}
  \item Plot an effective (i.e., as measured) TCP state-transition diagram for
    the two directions of a single TCP connection: states will be nodes, and
    transitions will be edges.
    Where state transitions diverge between the two directions, be sure to
    label edges indicating `client' vs. `server'.

  \item Extend the diagram to indicate, for each edge, the TCP header flags
    of the received packet triggering the transition, or the local system call
    (or other event -- e.g., timer) that triggers the transition.

  \item Compare the graphs you have drawn with the TCP state diagram in RFC
    793.

  \item Using DUMMYNET, explore the effects of simulated latency at 5ms
    intervals between 0ms and 40ms.
    What observations can we make about state-machine transitions as latency
    increases?
\end{enumerate}

\noindent
Be sure, in your lab report, to describe any apparent simulation or probe
effects.

\end{document}
