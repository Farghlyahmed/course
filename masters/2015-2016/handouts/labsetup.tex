\documentclass[a4paper,10pt]{article}
\usepackage{fullpage}
\usepackage{times}
\usepackage{upquote}
\usepackage{url}

\begin{document}

\newcommand{\code}[1]{\texttt{\small #1}}

\title{L41: Lab Setup}
\author{Dr Robert N.M. Watson}
\date{Michaelmas Term 2015}
\maketitle

L41 is taught through a blend of lectures and laboratory experiments.
The purpose of the labs is threefold: to teach you about real-world operating
systems, to teach you experimental methodology and practical skills, and to
provide fodder for assessment.
You will use tools such as DTrace to explore the behaviour of the system
through `potted' example programs that will trigger OS behaviours for you to
investigate.

Each lab is structured in two parts: a set of optional exploratory questions
to help you learn about OS performance and performance analysis, and a set of
mandatory experimental questions that you will address in your lab report.
The former are one possible way to approach the problem; you should feel free
to use other techniques or approaches.
Do ensure, however, that your lab report does address all of the assigned
experimental questions.

\section*{Experimental platform}

Our experimental platform is the open-source FreeBSD operating system running
on the BeagleBone Black (BBB) board, described in the remainder of this
handout.

\subsection*{The operating system: FreeBSD}

We will be using the open-source FreeBSD operating system's ARMv7 port on the
BeagleBone Black.
FreeBSD is of particular interest due to its tight integration of a number of
tracing and measurement tools (e.g., DTrace) and that it is built by default
with the Clang/LLVM compiler suite, which make it easier to insert additional
instrumentation.
You can learn more about FreeBSD by visiting the FreeBSD Project's website:

\smallskip
\url{https://www.FreeBSD.org/}
\smallskip

The course text, \textit{The Design and Implementation of the FreeBSD
Operating System, Second Edition} will be a useful reference, covering
concepts such as the process model, inter-process communications, and
filesystems.
There is also a section on the implementation of DTrace, which may be useful
background material for the labs.
\textit{DTrace: Dynamic Tracing in Oracle Solaris, Mac OS X and FreeBSD} is a
detailed, user-facing guide on how to use DTrace, and will also be a useful
reference for the labs.

\subsection*{The board: BeagleBone Black}

The BeagleBone Black board is based on a Texas Instruments System on Chip that
employs a single-core ARM Cortex-A8 processor; it includes on-board flash, an
Ethernet MAC, USB support, and an SD Card slot.
This 32-bit processor is not the zippiest in the world -- but it is widely
deployed and fully functional with respect to many of the behaviours of
interest for this course, including processor performance counters.
In our configuration, FreeBSD boots off of the on-board SD Card.

We will access the board as a USB device from lab workstations or personal
notebook computers.
``USB target mode'' allows the BBB to appear to be a set of USB devices to the
attached workstation: an Ethernet device that can be used to SSH into the
board, and a USB device that accesses a serial console.
Power is also provided using the USB cable, although we also have external
power adapters if you would prefer (i.e., if you want to leave the board
running an experiment while you unplug your notebook from it).
These instructions describe how to access the board from a Linux workstation
in the lab.
Other configurations, such as Mac or Windows notebooks, or FreeBSD
workstations, may also work with suitable adjustment to client-side tools.

While the BBB is not enormously expensive, they are often back ordered, so
acquiring replacements can be inconvenient.
If you lose or damage your board (e.g., by dropping it), then we may ask you
to pay to replace the board.
It is fairly straight forward to ``brick'' the board by overwriting the
contents of its on-board flash, the OS image on the SD Card, etc.; this is
discouraged.
If this appears to have occurred, please contact the module instructor for
assistance.
A small number of replacement boards are available if required.

It is easy to replace the SD Card in the board -- should it appear to have
become corrupted or entered an unusable state (they appear to be sensitive to
power being removed from the device while an operation is in progress), let us
know and we can provide a replacement.
Note that any customisation to your board (e.g., SSH keys, \url{/data}
partition) will not be present on the replacement card -- do ensure that you
keep the primary copy of any important data on your workstation.

\subsection*{Getting started}

The BeagleBone Black will arrive in a compact plastic case, and with a
pre-initialised SD Card holding the OS image.
For the purposes of this course, you should not need to disassemble the case
or eject the card, except to install an image upgrade provided by the
instructors.
We will power and communicate with the board via a single USB cable.
These directions assume that you will be working on an ACS Workstation running
Ubuntu Linux.
However, our teaching setup has also been used with varying degrees of success
with other systems, including FreeBSD and Mac OS X\footnote{We have had mixed
experience with Mac OS X and Windows, as the OS image on the BBB implements
`USB target mode' -- i.e., appears to be a set of USB devices -- and device
drivers on operating systems vary quite a bit.
If you appear to have reliability problems with the serial console or Ethernet
parts, it may well be a device-driver bug.
You are welcome to help us debug the problem, but it may be simpler to work
with the systems that we have tested with.}.
The BBB appears to the workstation operating system as two USB devices: a USB
serial port, which reaches a console, and a USB Ethernet device, which can be
used to SSH into the OS instance on the board.

When you plug in the BeagleBone Black, at least one blue LED should light up
on the board.
It can take a minute or so for the OS image to boot; once it does, the two USB 
devices will become visible on the workstation.
You can connect to the console using the following command:

\begin{small}
\begin{verbatim}
minicom -o -D /dev/ttyACM0
\end{verbatim}
\end{small}

You should be able to log in as \code{guest} without a password.
However, to log in via SSH, which will be much more convenient for streaming
data or transferring files, you will need to set up an \code{authorized\_keys}
file in \code{guest}'s \url{~/.ssh} directory, (\url{/data}).
If you already have a \url{~/.ssh/id_rsa.pub} (or similar) file in your CL
home directory, you can place that in the
\url{authorized_keys} file.
Otherwise, you can use \code{ssh-keygen} to generate a keypair first.
Once the key is installed, you should be able to SSH to the BBB:

\begin{small}
\begin{verbatim}
ssh guest@192.168.141.100
\end{verbatim}
\end{small}

To run \code{dtrace}, you will need to switch to the \code{root} user's
account using the \code{su} command, after which you should have a root shell
prompt.
When running as \code{root}, you should exercise caution: accidentally
deleting system files may (will) render your board unbootable or cause loss of
your scripts or data.
Now, run your first DTrace script:

\begin{small}
\begin{verbatim}
dtrace -qn 'BEGIN { printf("Hello world"); exit(0); }'
\end{verbatim}
\end{small}

To write the output of a script to a file, you can redirect standard output to
a file:

\begin{small}
\begin{verbatim}
cd /data
dtrace -qn 'BEGIN { printf("Hello world"); exit(0); }' > data.out
\end{verbatim}
\end{small}

We have configured the root filesystem to be read-only by default, in order to
reduce the chance of accidents, but also to reduce wear on the SD Card.
The \url{/data} filesystem is writable, however, and is a good place to stick
your scripts, data, compiler output, etc., so that it will persist across
reboots.
You will likely wish to copy files from the BeagleBone Black to your
workstation for analysis -- e.g., to load it into a spreadsheet.
You may also want to edit scripts and source files on the workstation.
You can copy files back and forth using \code{scp}; this command can be run on
the Linux workstation to copy \code{data.out} to your current working
directory:

\begin{small}
\begin{verbatim}
scp guest@192.168.141.100:/data/data.out .
\end{verbatim}
\end{small}

%\subsection*{Connecting to the TTL serial console}
%
%Although it is unlikely you will require access to the boot console, this can
%be done via a FTDI-to-TTL adapter available from the module instructor.
%The three pins should be labeled and connected as follows:
%
%\medskip
%\begin{tabular}{llll}
%\hline
%  Colour & Label & Function & Pin \\
%\hline
%  Black & GND & Ground & 1 \\
%  Red & TXD & Transmit Data & 4 \\
%  Green & RCD & Receive Data & 5 \\
%\hline
%\end{tabular}
%\medskip
%
%\noindent
%These should be connected to 6-pin serial block on the BeagleBone Black.
%Pin 1 has a white dot next to it.
%The default TTL console speed is 115,200 bps; attempting other speeds may
%interrupt the boot or trigger a serial break to the kernel debugger.
%

\subsection*{Shutting down}

Simply powering off the BeagleBone Black (e.g., by unplugging it) will be
harmless in most situations, but could lead to data loss.
It is preferable to instead first run \code{shutdown -p now} to initiate a
software poweroff.
Once the system has shut down, the blue LED will turn off, and it is safe to
unplug the board.

\end{document}
