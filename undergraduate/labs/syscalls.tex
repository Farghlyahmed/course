\documentclass[pdftex,handout]{beamer}
\usetheme{metropolis}

\usepackage[english]{babel}
\usepackage[latin1]{inputenc}
\usepackage{times}
\usepackage[T1]{fontenc}
\usepackage{fancyvrb}
\usepackage{listings}
\begin{document}
\lstset{language=C, escapeinside={(*@}{@*)}, numbers=left,
  basicstyle=\tiny, showstringspaces=false, showspaces=false, showtabs=false}

\title{Lab 1: System Calls}
\author[shortname]{George V. Neville-Neil \and Robert N. M. Watson}

\begin{frame}
  \titlepage
\end{frame}

\begin{frame}[fragile]
  \frametitle{Setup}
  \begin{enumerate}
  \item Boot the virtual machine
  \item Log in as \verb|root| on the console
  \item Load the DTrace kernel modules
    \begin{itemize}
    \item \verb|kldload dtraceall|
    \end{itemize}
  \item Check that the modules have loaded
    \begin{itemize}
    \item \verb|kldstat|
    \end{itemize}
  \end{enumerate}
\end{frame}

\begin{frame}
  \frametitle{Writing Data}
  \begin{itemize}
  \item Writing data is very common
  \item Using echo, cat, and vi or Emacs to write
  \item Narrow your scope
  \end{itemize}
\end{frame}

\begin{frame}[fragile]
  \frametitle{System Call Lab Exercises}
  \begin{itemize}
  \item Count all the system calls in the freebsd module
  \item Find and show all the system calls used by sshd
  \item Use a D script to record data written by echo
  \item Construct a D script to measure the write() system call throughput
  \item Construct a D script to measure the write() system call latency
  \end{itemize}
\end{frame}

\end{document}

%%% Local Variables:
%%% mode: latex
%%% TeX-master: t
%%% End:

